\documentclass{article}
\usepackage{fullpage}
\usepackage{graphicx}
\usepackage{amsmath,amssymb,amsfonts}
\newcommand{\scaledgraph}[2]{\includegraphics[scale=#1]{graphs/#2.png}}
\newcommand{\graph}[1]{\scaledgraph{0.5}{#1}}
\newcommand{\centeredgraph}[1]{\begin{center}\graph{#1}\end{center}}
\newcommand{\inlinegraph}[1]{\scaledgraph{0.2}{#1}}
\newcommand\Sum{\texttt{sum}}
\newcommand\Int{\texttt{int}}
\newcommand\Float{\texttt{float}}
\newcommand\F{\texttt F}
\newcommand\T{\texttt T}
\renewcommand\L{\texttt L}
\newcommand\R{\texttt R}
\newcommand\Fail{\texttt{fail}}

\newcommand\form[4]{#1 \rightarrow #2, #3 \qquad \qquad #4}
\begin{document}
\section{Locks and constraints}
\subsection*{Sums}

This section is motivated by the following example. We are given
sales data for a number of stores in a company, with 
each store identified by its state. The regional managers 
of the company can view the sum of their own region's sales 
data and change that number, indicating a wish to increase future
sales. If the manager wishes to change the sales figures from
$n$ to $N$, each store's figures will be multiplied by $\frac N n$
to make the change.
Similarly, the CEO can see the total sum of national
sales and adjust that number accordingly.  

%% Put diagram here

However, the sales data
for an individual store or region may be fixed or \emph{locked},
meaning that its sales data may not be changed.

On Notation: the lenses given here all have intuitive semantics in the 
``get'' direction, but in the ``put'' direction there is some
ambiguity. Therefore the examples detail only ``put'' behavior.
The put function of a lens takes (1) some input, (2) the current state of
the data, (3) some constraints on the data, and outputs either
failure or a new state. This is modeled using arrows:
\[ \form {\text{input}} {\text{current state}} {\text{constraints}} {\text{output}}.\]
When the state is a list of length $n$ and the constraint
on the data is that certain elements of the list may not change
in the output, we represent this \emph{locking constraint} as a boolean list
$O_1,\ldots,O_n$ of length $n$, where $O_i=T$ means the $i$th element
is locked, and $O_i=F$ otherwise. 

More general constraints are modeled as sets of feasible output.

\subsubsection*{Flat Sums}

We start with a lens which takes a single list of numbers
to its sum.
The default case, assuming no locks on the elements of the list,
distributes $n-\sum_{i=1}^m x_i$ among the list in proportion to
each $x_i$.
\begin{align}
    \form {200} {[10,20,30,40]} {\F\F\F\F} {[20,40,60,80]}
\end{align}

In the presence of locks on $\{x_j \mid j \in J\}$, we produce
a list which is the result of fixing the locked entries and 
distributing $n-\sum_{i=1}^m x_i$ values among the $m-|J|$ locked
elements of the list. 
\begin{align}
    \form {200} {[10,20,30,40]} {\T\F\F\F} {[10,42.2,63.3,84.4]}
\end{align}
To keep the first element of the list locked, we must scale
the other elements by a factor of $\frac {19} 9$. 

If the constraints do not allow the elements of the list
to be mutated to satisfy the sum, there are two possibilities
of output. The first option is to fail if the list cannot be mutated in-place.
\begin{align}
    \form {200} {[10,20,30,40]} {\T\T\T\T} {\Fail}
\end{align}
The second option is to add elements to the list so that the sum holds.
\begin{align}
    \form {200} {[10,20,30,40]} {\T\T\T\T} {[10,20,30,40,100]} 
\end{align}
In what context would this notion of sum be appropriate?

\subsubsection*{Split Sum}

Next we consider the lens which maps an ordered pair to
a labeled list, with each element of the list labeled 
as either \L (left) or \R (right). The left-labeled
elements of the list should sum up to the left element
of the ordered pair, and vice verse.

The default case with no locks behaves like an extension of the sum lens
above.
\begin{align}
    \form {(50,50)} {[\L10,\L20,\R30,\R40]} {\F\F\F\F} 
          {[\L16.67,\L33.33,\R21.43,\R28.57]}
\end{align}

Similarly, with the case for locks\ldots
\begin{align}
    \form {(50,50)} {[\L10,\L20,\R30,\R40]} {\T\F\F\T} 
          {[\L10,\L40,\R10,\R40]}
\end{align}

If locks prevent us from mutating either section of
the list to fit the constraint, we default to the options 
presented for plain sums in this situation.
\begin{align}
    \form {(50,70)} {[\L10,\L20,\R30,\R40]} {\T\T\F\F} {\Fail}
\end{align}
\begin{align}
    \form {(50,70)} {[\L10,\L20,\R30,\R40]} {\T\T\F\F} 
          {[\L10,\L20,\L20,\R30,\R40]}
\end{align}

\subsubsection*{Multi-Level Lenses}
We can represent the full sales example described in the beginning of
this section as a lens with multiple levels. From the CEO's point of
view, this lens stretches between the total sales data and a labeled
list of store sales data. 

When neither intermediary region is completely locked, a change from
$n$ to $N$ first scales up the intermediary regions by a factor of
$N/n$. The changes to each individual region must be accounted for,
even if some of its sub-elements are locked.
\begin{align}
    \form {200} {(30,70),[\L10,\L20,\R30,\R40]} {\T\F\F\F} 
          {(60,140),[\L10,\L50,\R60,\R80]}
\end{align}

When an entire region is locked however, we must propagate the locks
up to the regional level.
\begin{align}
    \form {200} {(30,70),[\L10,\L20,\R30,\R40]} {\T\T\F\F}
          {(30,170),[\L10,\L20,\R72.9,\R97.1]}
\end{align}

\subsection*{Cumulative Average}

A student is given some number of grades in a class, and she wants
to know what grades she needs to get in the future to obtain 
a desired average. The constraints on this problem may go beyond
locking to illuminate different environments. 

Given that all grades fall within a certain range, how many assignments
must the student complete to raise her average?
\begin{align}
    \form {95} {[80,100]} 
    {\{\text{lists like $[80,100,x_0,x_1,x_2,\ldots]$ where
      $0 \le x_i \le 100$}\}}
    {[80,100,100,100]}
\end{align}

The constraint might limit the number of future assignments.
\begin{align}
    \form {95} {[80,100]} 
    {\{\text{lists like $[80,100,x_0,\ldots,x_{n-1}]$ where
      $0 \le x_i \le 100$ and $n\le1$}\}}
    {\Fail}
\end{align}

\subsection*{Partition}

Constraints on the partition lens can be stated in a very general form,
and might be able to guide the lens's alignment/update policy. 
Let $A$, $B$ and $C$ be the following lists:
\begin{center} \begin{tabular}{l l | l | l}
    $A$ & &$B$ &$C$ \\
    \hline
    \L &Bach &Bach &Asimov\\
    \R &Asimov &Beethoven &Austen\\
    \L &Beethoven & &
\end{tabular} \end{center}
For example, we can use the constraint to model the structure
of the merge.
\begin{align}
    \form {B,C} {A} {\{\text{lists with authors first}\}} {~} 
    \begin{tabular}{l l}
        \R &\text{Asimov} \\
        \R &\text{Austen} \\
        \L &\text{Bach} \\
        \L &\text{Beethoven}
    \end{tabular}
\end{align}
\begin{align} 
    \form {B,C} {A} {\{\text{lists like $A\texttt{++}Z$
           where elements in $Z$ do not appear in $A$}\}} {~} 
    \begin{tabular}{l l}
        \L &\text{Bach} \\
        \R &\text{Asimov} \\
        \L &\text{Beethoven} \\
        \R &\text{Austen} 
    \end{tabular}
\end{align}

\subsection*{Database Operations}

In certain contexts, it might be valuable to view operations on relational
databases as lenses. We can represent get functions by SELECT INTO statements
and put functions by UPDATE statements, which take the old state of the
table into account. WHERE clauses represent constraints we have on the lenses. 

\subsubsection*{Select}

Let's start with a simple SELECT statement example.
Let \textsc{jobs} be the following table, describing the jobs of employees
at a company.
\begin{center} \begin{tabular} {c | c c c}
    \textsc{jobs} & id & name & title \\
    \hline
    & 1 & Adam Aardvark & CEO \\
    & 2 & Brian Badger & VP of Sales \\
    & 3 & Charlie Camel & VP of Engineering \\
    & 4 & Diane Duck & Software Engineer
\end{tabular} \end{center}
A simple select statement picks out the name of the CEO.
\begin{quote}
    SELECT name \\
    INTO \textsc{ceo} \\
    FROM \textsc{jobs} \\
    WHERE title=`CEO'
\end{quote}
This produces a table \textsc{ceo} as follows:
\begin{center} \begin{tabular} {c | c}
    \textsc{ceo} & name\\
    \hline
    & Adam Aardvark
\end{tabular} \end{center}
Suppose Adam gets married and wants to change his name to a
hyphenated version. The following query updates the \textsc{ceo} table:
\begin{quote}
    UPDATE \textsc{ceo} \\
    SET name=Adam Aardvark-Albatross
\end{quote}
The result is the following updated table:
\begin{center} \begin{tabular} {c | c}
    \textsc{ceo} & name\\
    \hline
    & Adam Aardvark-Albatross
\end{tabular} \end{center}
The putback function uses the same constraint to update the \textsc{jobs}
table as it used to select into the \textsc{ceo} table. 
\begin{quote}
    UPDATE \textsc{jobs} \\
    SET name = \textsc{ceo}.name \\
    WHERE title = `CEO' 
\end{quote}
The result is the following:
\begin{center} \begin{tabular} {c | c c c}
    \textsc{jobs} & id & name & title \\
    \hline
    & 1 & Adam Aardvark-Albatross & CEO \\
    & 2 & Brian Badger & VP of Sales \\
    & 3 & Charlie Camel & VP of Engineering \\
    & 4 & Diane Duck & Software Engineer
\end{tabular} \end{center}

\subsubsection*{Join}

Consider the JOIN operation as a function from a pair of tables to a single
table. We might also want to move from a single table back to the 
separate pairs. Let's start with two tables, \textsc{jobs} as before,
and \textsc{bosses} which gives the id numbers of employees and their bosses.
\begin{center} \begin{tabular} {c | c c c}
    \textsc{jobs} & id & name & title \\
    \hline
    & 1 & Adam Aardvark & CEO \\
    & 2 & Brian Badger & VP of Sales \\
    & 3 & Charlie Camel & VP of Engineering \\
    & 4 & Diane Duck & Software Engineer
\end{tabular} \end{center}
\begin{center} \begin{tabular} {c | c c }
    \textsc{bosses} & id & boss\_id \\
    \hline
    & 1 & 1 \\
    & 2 & 1 \\
    & 3 & 1 \\
    & 4 & 3 
\end{tabular} \end{center}
Using the JOIN operation, we can write the following query which lists 
the names of employees along with the names of their boss.
\begin{quote}
    SELECT J1.name, J2.name \\
    INTO \textsc{named\_bosses} \\
    FROM \textsc{jobs} J1 JOIN \textsc{bosses} B JOIN \textsc{jobs} J2 \\
    ON J1.id = B.id AND B.boss\_id=J2.id
\end{quote}
The result is the following:
\begin{center} \begin{tabular} {c | c c }
    \textsc{named\_bosses} & name & boss\_name \\
    \hline
    & Adam Aardvark & Adam Aardvark \\
    & Brian Badger & Adam Aardvark \\
    & Charlie Camel & Adam Aardvark \\
    & Diane Duck & Charlie Camel
\end{tabular} \end{center}
If we change Diane Duck's boss to Brian Badger, we get
\begin{center} \begin{tabular} {c | c c }
    \textsc{named\_bosses} & name & boss\_name \\
    \hline
    & Adam Aardvark & Adam Aardvark \\
    & Brian Badger & Adam Aardvark \\
    & Charlie Camel & Adam Aardvark \\
    & Diane Duck & Brian Badger
\end{tabular} \end{center}
The change needs to be reflected back into the \textsc{bosses}
table (as, in this case, no change is made in \textsc{jobs}).
\begin{quote}
    UPDATE \textsc{bosses} \\
    FROM \textsc{jobs} J1 JOIN \textsc{named\_bosses} N JOIN \textsc{jobs} J2 \\
    ON J1.name = N.name AND N.boss\_name = J2.name \\
    SET B.boss\_id = J2.id \\
    WHERE B.id = J1.id
\end{quote}
This update results in the expected \textsc{bosses} table:
\begin{center} \begin{tabular} {c | c c }
    \textsc{bosses} & id & boss\_id \\
    \hline
    & 1 & 1 \\
    & 2 & 1 \\
    & 3 & 1 \\
    & 4 & 2 
\end{tabular} \end{center}


\section{Graph selection}
\subsection*{Setting: Simplified Facebook}
Graphs have three kinds of nodes: user identity nodes, friends list nodes,
and name nodes. User identities point to a name node and a friends list
node; friends list nodes point to user identity nodes. For example:

\centeredgraph{facebook-start}

\subsubsection*{Choosing the name field}
The simplest thing you might want to do is pick out a particular person's
name. For example, let's play with a lens that we will call ``1:name'' for
now that picks out the node with user identity 1 and its corresponding name
node. In the get direction, the above graph would be in sync with the graph
\inlinegraph{facebook-name-dw}.

We could put back the graph \inlinegraph{facebook-name-dmw} to get the
graph:

\centeredgraph{facebook-start-dmw}

If we put back a graph like \inlinegraph{facebook-name-jp} with a name that
happens to already be in the graph, we would not want to introduce spurious
sharing:

\begin{center}
    \graph{facebook-start-jp} not \graph{facebook-spurious-sharing}
\end{center}

\subsubsection*{Choosing the friends field}

Let's call this lens ``3:friends:*''; it should behave pretty similarly to the
last one, picking out PMO's friends. For example, the starting graph would
be in sync with the graph \inlinegraph{facebook-friends}. However, unlike
the previous lens, if we put back a graph like
\inlinegraph{facebook-friends-1}, we should get some sharing:

\centeredgraph{facebook-start-more-friends}

Perhaps either the graph itself or some part of the lens should indicate
where sharing is desired and where not. For example, you might imagine
having two separate ``chunks'' of each graph: the chunk where sharing is
expected and nodes have permanent node identities, and chunks that are
tree-like attached to identified nodes.

\subsubsection*{Creating and deleting users}

It's not clear exactly what interface you want for this. You could keep a
strict analogy with lenses as we know them. In that case, you would want to
be able to write a lens, let's call it ``users'', which would pick out all
the roots of the graph. For example, the starting graph above would be in
sync with \inlinegraph{facebook-roots}. You would then create a new user by
putting back a graph like \inlinegraph{facebook-more-roots}, which would
result in something like:
\centeredgraph{facebook-start-more-roots}
Alternatively, you could put back a graph like
\inlinegraph{facebook-fewer-roots} to delete a user and (presumably) garbage
collect its fields:
\centeredgraph{facebook-start-fewer-roots}
Perhaps one would like some additional operation for creation and deletion,
though.

\subsection*{Setting: Simplified Github}

Graphs have three kinds of nodes: user nodes, project nodes, and names. We
will expect the user and project nodes to have identities, and therefore
this part of the graph should exhibit sharing during updates, whereas the
names should not. Here's a sample graph, using dotted lines for the
``unshared'' chunks of the graph:

\centeredgraph{github-start}

\subsubsection*{People who are working together}

We might want to pick out all of JP's colleagues. Let's call the lens that
does this ``u2:project:project''. The sample graph above would be in sync
with the graph:
\centeredgraph{github-colleagues}
We could then indicate that JP is no longer working on the crash project by
putting back the graph \inlinegraph{github-fewer-projects}, which would
result in:
\centeredgraph{github-start-fewer-projects}
This graph only differs from the original by the deletion of a project edge
between u2 and p1.

\subsubsection*{Adjusting names/friendship}

As with the simplified Facebook example, we expect that changing names
doesn't produce sharing but changing local friendship subgraphs might; so we
distinguish between these subgraphs using the solid and dotted edges.

\subsubsection*{Creating or deleting users and projects}

The discussion of the previous section applies almost verbatim: we could try
to cast this as a classical lens by creating one which picks out just the
user nodes or just the project nodes, but even after doing that it may make
sense to have a separate operation for creation (since, after all, it should
probably be the system's job to pick a globally unique ID for the new user
or project).

\section{Many-directional programming}
\subsection*{Net sales}
This example is as discussed at LogicBlox: we have sales, costs, and net
sales. Each are lists, and the sales are the pointwise sum of the net sales
and the costs. To begin with, we have three repositories; throughout this
document, we'll label repository states with capital letters. Thus:
\begin{align*}
    S &= [110,220,330,440] \\
    C &= [100,200,300,400] \\
    N &= [10,20,30,40]
\end{align*}
These are our original, synchronized repositories.

For updates to the net sales $N$ or the costs $C$, it seems fairly clear
what should happen.

\[N' = [10,20,30,50] \rightarrow (S,C) \Rightarrow S' = [110,220,330,450]\]
\[C' = [100,200,500,400] \rightarrow (S,N) \Rightarrow S' = [110,220,530,440]\]

However, if we update $S'=[110,220,450,440]$, there seem to be more choices.
Perhaps one natural choice is to scale $N$ and $C$ together:
\begin{align*}
    N' &= [10,20,\frac{450}{300}30,40] \\
    C' &= [100,200,\frac{450}{300}300,400]
\end{align*}
However, we could also imagine shifting everything into $N$ or into $C$:
\[N' = [10,20,180,40]\]
or:
\[C' = [100,200,450,400]\]
or distribute the extra evenly:
\begin{align*}
    N' &= [10,20,105,40] \\
    C' &= [100,200,375,400]
\end{align*}

Perhaps each of these is reasonable at different times.

\subsection*{GUI}
We might imagine designing a text editor, which has three widgets: a text
buffer, the displayed text, and a scroll bar state containing the current
chunk being displayed and the total number of chunks.
\begin{align*}
    A &= [some,long,text] \\
    B &= long \\
    C &= (2,3)
\end{align*}

For some updates, it is clear what to do
\begin{itemize}
    \item If we set $C'=(3,3)$ (scrolling down), we should update $B' =
        text$.
    \item If we set $B'=short$, we should update $A=[some,short,text]$.
    \item If we update $A'=[no,long,text]$, $B$ and $C$ may remain as-is.
    \item If we update $A'=[some,very,long,text]$, we should update
        $C'=(3,4)$.
\end{itemize}

However, if $A'$ is radically different from $A$, we have some choices to
make that require heuristics. Choices include keeping the current line
number, jumping to the beginning, jumping to the end, and keeping the current ``percentage of document''.

Moreover, it's not at all clear that these are the right data structures at
all: for example, it's not clear how to delete a line from $A$ just from the
$B$ widget.

Let $A$ be the following buffer:
\begin{align*}
  &\text{apple} \\
  &\text{banana} \\
  &\text{cherry} \\
  &\text{dog} \\
  &\text{eclair}
\end{align*}
and let us be focused on line 3, meaning that $B = \text{cherry}$ and $C = (3,5)$.
If we delete the third line, changing $A$ to
\begin{align*}
  &\text{apple} \\
  &\text{banana} \\
  &\text{dog} \\
  &\text{eclair}
\end{align*}
then we must refocus the $B$ and $C$ views on another line. 
The following options are reasonable outputs to this operation:
\begin{center}\begin{tabular}{c c c}
  Description & $B$ & $C$ \\
  \hline
  Jump to the beginning & apple & $(1,4)$ \\
  Jump to the end & eclair & $(4,4)$ \\
  Jump to the next line & dog & $(3,4)$ \\
  Jump to the previous line & banana & $(2,4)$ \\
  Jump to the (round $\frac 3 5 4$)th line & banana & $(2,4)$
\end{tabular}\end{center}

While the ``next line'' and ``previous line'' options make sure
the edit buffer stays in the same general area in the file, the
last option allows the edit buffer to scale through the buffer.
This effect becomes more prominent if the $B$ buffer holds
more than one line. 

On the other hand, what if our goal is to delete the line given
by $B$ and propagate the update to $A$ and $C$? When we update
$B = <\texttt{deleted}>$, we can change $A$ to be the desired
\begin{align*}
  &\text{apple} \\
  &\text{banana} \\
  &\text{dog} \\
  &\text{eclair}.
\end{align*}
But what do we expect for $C$? In a standard
text editor, we would expect the cursor to jump to the next line,
and then fill the $B$ buffer with that line. But our lenses
likely would not accommodate that kind of feedback.
Therefore the only possible output
for $C$ is a ``non-existent'' marking, like $(-1,4)$. 

\subsection*{Function focusing}

For this example we have three structures: 
$A$ contains a function $f : \alpha \rightarrow \beta$;
$B$ contains an input $b : \alpha$; and 
$C$ contains some output $c : \beta$. The
lens is in sync if $f(b) = c$. 

If we change the value of $f$ to $f':\alpha \rightarrow \beta$,
we must update $C$ to $f'(b)$ in order to stay in sync. 

If we change $B$ to some $b' : \alpha$, we have two options. 
First, we might update $C$ to contain $f(b')$; this option
represents a look-up. 
Second, we might change $A$ to refer to the function 
\[ \lambda v. \text{ if } v=b' \text{ then } c \text{ else } f~v. \]

Similarly, if we change $C$ to some $c' : \beta$, we should update $A$ to 
refer to 
\[ \lambda v. \text{ if } v=b \text{ then } c' \text{ else } f~v. \]

\subsection*{Wikipedia structure}

The next example has the property that one of the three structures encompass
all the information held by the other two. 

Wikipedia articles include the following three parts: $A$, the source which is
a list of headings $h_i$, content $c_i$ and authors $a_i$; $B$, the table of
contents which consists of headings and authors; and $C$, the article body
which consists of headings and content. The layout is summarized as follows:
\begin{center} \begin{tabular}{c c c}
    Source & Table of Contents & Body \\
    \hline
    $h_1,c_1,a_1$ & $h_1,a_1$ & $h_1,c_1$ \\
    $h_2,c_2,a_2$ & $h_2,a_2$ & $h_2,c_2$ \\
    $h_3,c_3,a_3$ & $h_3,a_3$ & $h_3,c_3$ 
\end{tabular} \end{center}
Every update to the source $A$ can be propagated to $B$ and $C$ by
simple projections. On the other hand, updates to $B$ can be
put into $A$ and then projected from $A$ to $C$; the same holds
for updates to $C$. Therefore, every operation by this lens
can be seen as a combination of get and putback functions 
between $A \leftrightarrow B$ and $A \leftrightarrow C$. 

\subsection*{Partition}

We can view the partition lens perhaps more naturally as a lens
between three data structures: 
\begin{align*}
  A \subset (X + Y)^\ast \qquad B \subset X^\ast \qquad C \subset Y^\ast
\end{align*}
The behavior is almost exactly the same as the traditional partition
problem. Updates to $A$ project to $B$ and $C$ as expected. Updates to
$B$ affect $A$ but not $C$, and vice verse. The only caveat is
that while the traditional partition can accommodate simultaneous
updates to $B$ and $C$, the three-way version can only update
$B$ and $C$ independently. 

\subsection*{List concatenation} 

To motivate list concatenation, suppose we have three data structures:
$A$ which records a first name, $B$ which records a last name, and
$C$ which is the concatenation of $A$ and $B$. Changes to $A$ and $B$
can easily result in an update to $C$. The interesting point is when
an update is made to $C$ which adds a word between the first and
last name. Since that insertion may be a first name or a last name,
the separation again is ambiguous. 

As an example, consider the names Sandy \emph{dos} Santos\footnote{dos Santos
is a popular Portuguese and Spanish last name.}
and Junie \emph{B.} Jones\footnote{The protagonist of a children's
book series, Junie B. gets mad at anyone who refuses to call her by her chosen
first name.}


\end{document}
