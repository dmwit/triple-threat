\documentclass{article}
\usepackage{graphicx}
\newcommand{\scaledgraph}[2]{\includegraphics[scale=#1]{graphs/#2.eps}}
\newcommand{\graph}[1]{\scaledgraph{0.5}{#1}}
\newcommand{\centeredgraph}[1]{\begin{center}\graph{#1}\end{center}}
\newcommand{\inlinegraph}[1]{\scaledgraph{0.2}{#1}}
\begin{document}
\section{Setting: Simplified Facebook}
Graphs have three kinds of nodes: user identity nodes, friends list nodes,
and name nodes. User identities point to a name node and a friends list
node; friends list nodes point to user identity nodes. For example:

\centeredgraph{facebook-start}

\subsection{Choosing the name field}
The simplest thing you might want to do is pick out a particular person's
name. For example, let's play with a lens that we will call ``1:name'' for
now that picks out the node with user identity 1 and its corresponding name
node. In the get direction, the above graph would be in sync with the graph
\inlinegraph{facebook-name-dw}.

We could put back the graph \inlinegraph{facebook-name-dmw} to get the
graph:

\centeredgraph{facebook-start-dmw}

If we put back a graph like \inlinegraph{facebook-name-jp} with a name that
happens to already be in the graph, we would not want to introduce spurious
sharing:

\begin{center}
    \graph{facebook-start-jp} not \graph{facebook-spurious-sharing}
\end{center}

\subsection{Choosing the friends field}

Let's call this lens ``3:friends:*''; it should behave pretty similarly to the
last one, picking out PMO's friends. For example, the starting graph would
be in sync with the graph \inlinegraph{facebook-friends}. However, unlike
the previous lens, if we put back a graph like
\inlinegraph{facebook-friends-1}, we should get some sharing:

\centeredgraph{facebook-start-more-friends}

Perhaps either the graph itself or some part of the lens should indicate
where sharing is desired and where not. For example, you might imagine
having two separate ``chunks'' of each graph: the chunk where sharing is
expected and nodes have permanent node identities, and chunks that are
tree-like attached to identified nodes.

\subsection{Creating and deleting users}

It's not clear exactly what interface you want for this. You could keep a
strict analogy with lenses as we know them. In that case, you would want to
be able to write a lens, let's call it ``users'', which would pick out all
the roots of the graph. For example, the starting graph above would be in
sync with \inlinegraph{facebook-roots}. You would then create a new user by
putting back a graph like \inlinegraph{facebook-more-roots}, which would
result in something like:
\centeredgraph{facebook-start-more-roots}
Alternatively, you could put back a graph like
\inlinegraph{facebook-fewer-roots} to delete a user and (presumably) garbage
collect its fields:
\centeredgraph{facebook-start-fewer-roots}
Perhaps one would like some additional operation for creation and deletion,
though.

\section{Setting: Simplified Github}

Graphs have three kinds of nodes: user nodes, project nodes, and names. We
will expect the user and project nodes to have identities, and therefore
this part of the graph should exhibit sharing during updates, whereas the
names should not. Here's a sample graph, using dotted lines for the
``unshared'' chunks of the graph:

\centeredgraph{github-start}

\subsection{People who are working together}

We might want to pick out all of JP's colleagues. Let's call the lens that
does this ``u2:project:project''. The sample graph above would be in sync
with the graph:
\centeredgraph{github-colleagues}
We could then indicate that JP is no longer working on the crash project by
putting back the graph \inlinegraph{github-fewer-projects}, which would
result in:
\centeredgraph{github-start-fewer-projects}
This graph only differs from the original by the deletion of a project edge
between u2 and p1.

\subsection{Adjusting names/friendship}

As with the simplified Facebook example, we expect that changing names
doesn't produce sharing but changing local friendship subgraphs might; so we
distinguish between these subgraphs using the solid and dotted edges.

\subsection{Creating or deleting users and projects}

The discussion of the previous section applies almost verbatim: we could try
to cast this as a classical lens by creating one which picks out just the
user nodes or just the project nodes, but even after doing that it may make
sense to have a separate operation for creation (since, after all, it should
probably be the system's job to pick a globally unique ID for the new user
or project).

\end{document}
