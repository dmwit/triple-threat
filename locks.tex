\documentclass{article}
\usepackage{fullpage}
\usepackage{amsmath,amssymb,amsfonts}

\title{Lens Examples for Locks and Constraints}

\begin{document}
\maketitle

\newcommand\Sum{\texttt{sum}}
\newcommand\Int{\texttt{int}}
\newcommand\Float{\texttt{float}}
\newcommand\F{\texttt F}
\newcommand\T{\texttt T}
\renewcommand\L{\texttt L}
\newcommand\R{\texttt R}
\newcommand\Fail{\texttt{fail}}

\newcommand\form[4]{#1 \rightarrow #2, #3 \qquad \qquad #4}

\subsection{Sums}

\subsubsection{Flat Sums}
A sum lens takes a list of numbers to its sum. 
\begin{align*}
    \Sum : \Float \leftrightarrow \Float^\ast
\end{align*}
Given a list, the put function produces the sum 
of the numbers in the list; given an integer
and a current list, the put function produces
a new list which sums up to the input.

In these examples, the put function has an additional
argument, a list of booleans representing which elements
of the list of numbers should be locked. 
The canonical sum with no locks is as follows:
\[ \text{Input} \qquad \qquad \qquad \qquad \qquad \text{Output} \]
\[ \form {200} {[10,20,30,40]} {\F\F\F\F} {[20,40,60,80]} \]
The difference between the sum of $[10,20,30,40]$ and the input $200$
is $100$, so we scale every entry in the list by a factor of two. 
When one or more of the list elements is locked, the scaling factor
changes.
\[ \form {200} {[10,20,30,40]} {\T\F\F\F} {[10,42.2,63.3,84.4]} \]
To keep the first element of the list locked, the other elements
must now scale by a factor of $\frac {19} 9$. 

If all the elements in the list are locked, there are two options.
First is to fail if the list cannot be mutated to conform to the
sum.
\[ \form {200} {[10,20,30,40]} {\T\T\T\T} {\Fail} \]
The second option is to add elements to the list (changing
the original list as little as possible) so that the sum holds.
\[ \form {200} {[10,20,30,40]} {\T\T\T\T} {[10,20,30,40,100]} \]
It is unclear in which contexts this notion of sum is appropriate.

\subsubsection{Split Sum}

Next we consider the lens which maps an ordered pair to
a labeled list, with each element of the list labeled 
as either \L (left) or \R (right). The left-labeled
elements of the list should sum up to the left element
of the ordered pair, and vice versa. 
\[ \form {(50,50)} {[\L10,\L20,\R30,\R40]} {\F\F\F\F} {[\L16.67,\L33.33,\R21.43,\R28.57]} \]
\[ \form {(50,50)} {[\L10,\L20,\R30,\R40]} {\T\F\F\T} {[\L10,\L40,\R10,\R40]} \]
Again, if locks prevent us from mutating the list to fit the constraint, 
there are two options.
\[ \form {(50,70)} {[\L10,\L20,\R30,\R40]} {\T\T\F\F} {\Fail} \]
\[ \form {(50,70)} {[\L10,\L20,\R30,\R40]} {\T\T\F\F} {[\L10,\L20,\L20,\R30,\R40]} \]

\end{document}
